\documentclass[output=paper]{langsci/langscibook} 
\ChapterDOI{10.5281/zenodo.4449763}
\title{Preface} 
\author{Jorge Pinto\affiliation{Centro de Linguística da Universidade de Lisboa}\lastand Nélia Alexandre \affiliation{Centro de Linguística da Universidade de Lisboa}}  

\abstract{}

\begin{document}
\maketitle

\noindent In the last three decades, there have been several approaches, models and theories that have developed around the acquisition of a second language. Theories based on years of research in a wide variety of fields, including linguistics, psychology, sociology, anthropology and psycholinguistics (\citealt{FreemanFreeman2001}). However, none can cover all the needs inherent to the teaching\slash learning process \citep{Cook2001Second}, nor has it been possible to arrive at a unified or comprehensive view of how second languages are learned (\citealt{MitchelMyles2004}; \citealt{Nunan2001}). This complexity is, among others, due to the fact that there are variations in the context where the acquisition processes occur that influence the nature of the input as well as the learning strategies used by the student, and due to biological variations of students, such as age, aptitude and intelligence, motivation, personality, and cognitive styles \citep{Ellis1989}.

The acquisition of (several) second languages has become a subject even more complex with globalization, the growing learning of foreign languages and the increasing number of multilingual speakers. For that reason, since the beginning of this century, there has been a growth in interest in multilingualism and, consequently, a proliferation of studies on the acquisition of a third language or additional language (L3\slash L\textit{n}), highlighting the differences with respect to the acquisition of an L2 and setting themselves a new area of research (\citealt{Jessner1999,HerdinaJessner2000,Cook2001Second,Cenoz2003}). These researches emphasize the benefits of multilingual education and show how multilingual acquisition is processed. Studies in the area of L3\slash L\textit{n} have largely contributed to a better understanding of the phenomenon.

Following this movement, the purpose of this book is to present recent inquiries in the field of multilingualism and L3, bringing together contributions from an international group of specialists from Austria, Canada, Germany, Portugal, Spain, Switzerland, Turkey, and United States. The main focuses of the articles are two: language acquisition and language learning\slash teaching in multilingual settings. A collection of theoretical and empirical articles from scholars of multilingualism and language acquisition makes the book a significant resource for teachers and researchers as the articles present a wide perspective from main theories to current issues and reflects new trends in the field. Since the heterogeneity and complexity that characterize multilingual acquisition and learning/teaching is a field of research in fast development and with an increasing interest rate, we believe that the texts included here will be of great relevance for the scientific community. 

Part I of the volume presents different topics of L3 acquisition, such as phonology, working memory, and selective attention. Namely, in chapter one, Ghiselli presents an experimental study about working memory and selective attention of conference interpreters. The author seeks to prove the hypothesis that the time dedicated to the practice and the type of activities done during self-study would contribute to the improvement of working memory and selective attention.

In chapter two, Zhou, Freitas, and Castelo present the results of a research on the acquisition of some phonological properties that are problematic for Chinese learners of European Portuguese, namely, the developmental patterns of acquisition of the liquid consonants.

The texts included in Part II show how the research on language acquisition informs pedagogical issues. For instance, how the context of learning previous languages influences the teaching of L3 is addressed by Carvalho, in chapter three. This chapter explores the main differences between bilingual English-Spanish learners of Portuguese L3 who acquired Spanish as an L2 or as a heritage language, both in terms of their performance in the classroom and in terms of the perception of their learning process. Kucukali, in the next chapter, reports the attitudes of three Turkish multilingual teachers (speakers of English, German, and Russian) to plurilingual approaches, and the response of their L3/L\textit{n} students to the plurilingual practices in class, showing that teachers who speak diverse foreign languages play an important role in multilingual classrooms. Paquet and Woll, in chapter five, tackle the benefits of crosslinguistic pedagogy versus classroom monolingual bias, based on an experimental study comprising forty trainees, from Canada and Mexico. The study addresses in particular their perceptions about the use of L1 or other languages in the classroom and the reasons why they use them. Part II also includes a study conducted by Barnes and Almgren showing how to provide adequate tools for trainees to deal with multilingual classrooms in the Basque Country, starting by understanding their own multilingual skills.

Following the previous studies about acquisition and teaching, it is relevant to consider a section concerning language learning aspects. Therefore, Part III comprises texts on individual learning strategies, such as motivation and attitudes, crosslinguistic awareness, and students’ perceptions about teachers’ “plurilingual non-nativism”. Hofer, in chapter seven, addresses the dynamic and complex nature of multilingual development as well as the need for speakers’ interaction for language enhancing. In the subsequent chapter, Mayr puts forward the results of a qualitative study on the development of crosslinguistic awareness in multilingual learning settings, involving plurilingual task-based approach, carried out at a secondary school in South Tyrol, Bolzano. At the end of this part of the volume, Yanaprasart and Melo-Pfeifer compare students’ perceptions of non-native teachers’ discourses and their intelligibility. The authors try to answer three main questions: (i) how students perceive a “plurilingual teacher”? (ii) how their perceptions are discursively reported? and (iii) how are these perceptions related to the profile of institutions and disciplinary fields?

Finally, we must highlight that these contributions include several different languages in contact in an acquisition/learning context: Basque, English, French, German, Italian, Ladin, Mandarin Chinese, Portuguese, Russian, Spanish, and Turkish. All the topics covered in this book are scientifically relevant, serving as support to student-teachers, teachers, as well as to all researchers whose work focuses primarily on Multilingualism and Third Language Acquisition. Particularly in a world context where schools are less and less monolingual and monocultural, and where linguistic and cultural diversity is increasing, these studies can help teachers to better cope with these situations in the classroom and provide a valuable resource for researchers.

{\sloppy\printbibliography[heading=subbibliography,notkeyword=this]}
\end{document}

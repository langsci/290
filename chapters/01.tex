\documentclass[output=paper]{../langscibook}
\author{Serena Ghiselli\affiliation{University of Bologna}}
\title{Cognitive processes and interpreting expertise: Autonomous exercise of master’s students}
\abstract{The present paper describes the results of a part of a PhD project about working memory (\citealt{BaddeleyHitch1974}; \citealt{Baddeley2000}; \citealt{Gerver1975,Gerver1976}; \citealt{PadillaBenitez1995}) and selective attention (\citealt{Cowan2000}; \citealt{Moser-Mercer2000}; \citealt{Seeber2011}; \citealt{TimarovaEtAl2014}, 2015) in the training of conference interpreters. 

In this experimental study data about autonomous interpreting exercise were collected. The study group was formed of interpreting students of the master’s degree in interpreting of the University of Bologna and can be divided into two subgroups, one subgroup starting the course in 2015 (27) and the other in 2016 (22). The hypothesis was that time devoted to exercise and the type of activities done during self-study would contribute to the improvement of working memory and selective attention, which were measured by a battery of psychological tests.

Before the description of empirical data, the paper includes a review of the main studies on skill acquisition (\citealt{Ackerman1988}; \citealt{Anderson1995}; \citealt{Ericsson2000}; \citealt{BastianOberauer2014}) and on cognitive training methods in interpreter training (\citealt{vanDam1989}; \citealt{DollerupLoddegaard1992}; \citealt{PadillaBenitez2002}; \citealt{Gillies2013}; \citealt{AndresBehr2015}; \citealt{YenkimalekivanHeuven2013,YenkimalekivanHeuven2017}; \citealt{SettonDawrant2016a,SettonDawrant2016a}.

% \keywords{Interpreter training, interpreting exercise, skill acquisition, working memory, selective attention}
}

\IfFileExists{../localcommands.tex}{
  \addbibresource{../localbibliography.bib}
  % add all extra packages you need to load to this file

\usepackage{tabularx,multicol}
\usepackage{url}
\urlstyle{same}

\usepackage{enumitem}

\usepackage{pifont}

\usepackage{listings}
\lstset{basicstyle=\ttfamily,tabsize=2,breaklines=true}

\usepackage{./langsci-optional}
\usepackage{./langsci-lgr}
\usepackage{./langsci-gb4e}

\usepackage{langsci-plots} 

\makeatletter
\let\pgfmathModX=\pgfmathMod@
\usepackage{pgfplots,pgfplotstable}%
\let\pgfmathMod@=\pgfmathModX
\makeatother

\usepackage{siunitx}
\sisetup{output-decimal-marker={.},detect-weight=true, detect-family=true, detect-all, input-symbols={\%}, free-standing-units,table-align-text-pre=false,group-digits=false,detect-inline-weight=math}
\DeclareSIUnit[number-unit-product={}]{\percent}{\%}
\makeatletter \def\new@fontshape{} \makeatother
\robustify\bfseries % For detect weight to work

\usepackage{todonotes}

  \newcommand*{\orcid}{}

\renewcommand{\lsChapterFooterSize}{\footnotesize}

\makeatletter
\let\thetitle\@title
\let\theauthor\@author
\makeatother

\newcommand{\togglepaper}[1][0]{
%   \bibliography{../localbibliography}
  \papernote{\scriptsize\normalfont
    \theauthor.
    \thetitle.
    To appear in:
    Jorge Pinto \& Nélia Alexandre (eds.),
    Multilingualism and third language acquisition: Learning and teaching trends.
    Berlin: Language Science Press. [preliminary page numbering]
    }
  \pagenumbering{roman}
  \setcounter{chapter}{#1}
  \addtocounter{chapter}{-1}
}



  %% hyphenation points for line breaks
%% Normally, automatic hyphenation in LaTeX is very good
%% If a word is mis-hyphenated, add it to this file
%%
%% add information to TeX file before \begin{document} with:
%% %% hyphenation points for line breaks
%% Normally, automatic hyphenation in LaTeX is very good
%% If a word is mis-hyphenated, add it to this file
%%
%% add information to TeX file before \begin{document} with:
%% %% hyphenation points for line breaks
%% Normally, automatic hyphenation in LaTeX is very good
%% If a word is mis-hyphenated, add it to this file
%%
%% add information to TeX file before \begin{document} with:
%% \include{localhyphenation}
\hyphenation{
affri-ca-te
affri-ca-tes
au-ton-o-mous
Cha-basse
Din-ger-fel-der
plu-ri-lin-gual
Ya-na-pra-sart
Mi-ri-ci
Ström-quist
}

\hyphenation{
affri-ca-te
affri-ca-tes
au-ton-o-mous
Cha-basse
Din-ger-fel-der
plu-ri-lin-gual
Ya-na-pra-sart
Mi-ri-ci
Ström-quist
}

\hyphenation{
affri-ca-te
affri-ca-tes
au-ton-o-mous
Cha-basse
Din-ger-fel-der
plu-ri-lin-gual
Ya-na-pra-sart
Mi-ri-ci
Ström-quist
}

  \togglepaper[1]%%chapternumber
}{}


\begin{document}
\maketitle
\shorttitlerunninghead{Cognitive processes and interpreting expertise}


\section{Introduction}


Interpreting expertise is not a natural ability but a hard earned result achieved by individuals with an aptitude for interpreting and thanks to targeted and constant effort. The PhD project on which this paper is based originated from the idea that interpreters are made, so to become a professional interpreter specific training and constant practice are needed. The project focused on working memory (WM) (\citealt{BaddeleyHitch1974}, \citealt{Baddeley2000}, \citealt{Gerver1975,Gerver1976}, \citealt{PadillaBenitez1995}) and selective attention (\citealt{Cowan2000}, \citealt{Moser-Mercer2000}, \citealt{Seeber2011}, \citealt{TimarovaEtAl2014}; 2015), which were measured through a battery of psychological tests.

In the project there was a study group of students attending the master’s degree in interpreting and a control group of students attending the master’s degree in translation. Every group was divided into two subgroups: students who started the master’s degree in 2015 and students who started the master’s degree in 2016 (27 and 22 interpreting students respectively; 23 and 37 translation students respectively). For the students who started the master’s degree in 2015 data were collected over two academic years, whereas for the students who started in 2016 over one academic year due to the PhD program time constraints.

For interpreting students, in addition to psychological test results, data about self-study were collected. The focus of the present paper is on self-study data and the aim is describing autonomous exercise habits of interpreting students. These data were originally collected as part of the PhD project because they were considered to be a relevant variable that could influence psychological test results.

On the basis of bibliographical research, publications on interpreting student self-study are very scarce. Two studies included data collection about autonomous exercise (\citealt{Fan2012}, \citealt{Wang2016}). The picture that emerges is that targeted exercise is important to automatize interpreting practice as much as possible and that the quality of self-study is more relevant than its quantity.

The improvement of a specialised skill happens when individuals are motivated, receive feedback and can repeat training activities \citep{EricssonEtAl1993}. Starting from this assumption, data collection focused on the frequency and on the typology of exercises done by interpreting students. Data collection aimed also at favouring participation and, for this reason, the method chosen was an email containing a brief survey that was sent to students every month. The goal was collecting a sample of data that could represent study habits and also avoid that students dropped the study because it took too much of their time.~


\section{Theoretical framework}


\subsection{Skill acquisition and expertise}



An expert is somebody who has achieved a high level of performance and skill in a specific domain as a result of experience. \citet[290]{Ackerman1988} described skill acquisition as a continuous process during task practice. The process of skill acquisition involves the decline of cognitive load from novice attention-demanding processing to skilled automatic processing.

\citet{Anderson1995} developed the concept of what happens during skill acquisition by identifying three stages in this process: the cognitive stage, the associative stage and the autonomous stage. In the first stage novices develop declarative knowledge, that is they memorise a series of elements that are relevant for that ability. In the associative stage novices gradually identify and eliminate mistakes. In the autonomous stage, the procedures novices have learnt become more and more automatic. When a novice turns verbal and declarative knowledge into procedural knowledge the learning process is almost complete.

In the case of interpreting, a crucial part of the skill acquisition process is the proficient use of WM, the short-term memory that actively decodes and stores information during complex cognitive activities \citep{Baddeley1997}. This is a very important skill that, in the case of interpreters, needs to be trained to work at a high performance in a situation of cognitive load and stress and that has to be coordinated with an efficient attention-switching system in order to manage all the simultaneous activities involved in interpreting.



\subsection{Expertise in conference interpreting}
\label{sec:ghiselli:2.2}


One of the earliest studies about the skills that influence academic performance of interpreting students was carried out by \citet{GerverEtAl1984}. The scholars compared the results of 12 English and French tests taken by 29 students before starting an intensive course of interpreting with the results of their final exams. Tests included two recall exercises, in which students were asked to repeat two oral texts of 1000 words each without taking notes, a cloze test, that is completing missing words in English oral texts of 500 words each and an error-detection test, which involved the recognition of mistakes in an oral text. It was found that recall exercises predicted the differences in the performance of consecutive interpreting, whereas cloze tests predicted the differences in simultaneous interpreting. Generally speaking, test results were better for the students who passed final exams.

In the studies about aptitude for interpreting carried out at the Advanced Schools for Interpreters and Translators of Trieste and Forlì (\citealt{PippaRusso2002}, \citealt{RussoPippa2004}, \citealt{Russo2014}) it was found that on-line paraphrase is a predictor of academic success. On-line paraphrase is an exercise in which the student has to understand and report a text while listening to it and to produce another cohesive and coherent text which carries the same meaning using different words.

In many domains, such as sport and music, it was demonstrated that quantity and quality of solitary activities are essential to develop skills (\citealt{Ericsson1996,Ericsson2001,Ericsson2002}, \citealt{HelsenEtAl1998}). It is therefore reasonable that, also for interpreting, solitary activities influence performance. Only two research papers, as far as it is known, took into account the self-study habits of interpreting students: \citet{Fan2012} and \citet{Wang2016}.

The study of \citet{Fan2012} analysed the various factors that can influence the development of interpreting competence in a group of 30 Chinese mother tongue students of the University of Newcastle. Students answered a survey at the beginning, in the middle and at the end of the academic year. Factors taken into account included autonomous interpreting exercise. Students had to estimate the average time devoted to self-study daily, without specifying which type of exercise they did. The results were that students devoted one and a half hour to self-study daily over the first semester and one hour and 45 minutes in the second semester. No statistically significant relation between the time devoted to self-study and academic performance was found. From a regression model a positive relation between the use of learning strategies and academic performance was found. The level of English measured through the IELTS exam also had a positive and statistically significant relation with consecutive interpreting exam results. These findings underlined that more than the quantity of time devoted to exercise it was the quality that counted.

\citet{Wang2016} carried out a longitudinal study with three interpreting Chinese students. Data were collected through weekly diaries and monthly interviews. The results were that students did exercises at home every day and often in groups. This paper does not give quantitative data, but from the interviews it emerged that in oral comprehension students initially understood single words and then moved to the comprehension of the global meaning. They also learnt to take less notes and memorise more. This study supports the idea that cognitive processes are essential to develop interpreting competence, which is the result of targeted practice.

In comparison with the existing studies about autonomous exercise habits of interpreting students, the study presented in this paper collects more data over a longer period of time and for a higher number of participants. The study of \citet{Fan2012} takes into consideration only the time devoted to self-study but not the type of activity performed. The study of \citet{Wang2016}, instead, focuses only on qualitative data. The present study tries to combine both qualitative (type of activity) and quantitative data (frequency and duration) in order to provide a more detailed description of interpreting trainees’ study habits.



\subsection{Cognitive training exercises for interpreters}
\label{sec:ghiselli:2.3}


In interpreting studies literature there are some examples of exercises that can be done in class or that can be assigned for self-study. In the present paper the main exercises have been divided into three groups according to their function and will be briefly described. The groups are: exercises to improve memory, exercises to improve consecutive interpreting and exercises to improve simultaneous interpreting.

As regards the improvement of memory, \citet{SettonDawrant2016b} suggested that WM limits may be put forward if information is organised in a scheme and divided into significant units. In addition, processes such as discourse analysis, note-taking and language switching have to be automatized. The authors distinguished between discourse modelling and discourse outlining (\citealt{SettonDawrant2016a}). Discourse modelling is a generic term to indicate the process of shaping a mental model of a discourse. Discourse outlining is the act of writing a representation of this model through a list with bullet points. A discourse model is a mental model of the discourse created when listening to a text with the intention of memorising it and that helps to analyse and memorise information. The authors suggest doing a WM  exercise called \emph{idiomatic gist}. This exercise involves the recall of a short text (reading time: 30--45 sec.) having a sophisticated style, which forces the reader to go beyond the words and focus on the meaning behind them.

Most people remember better what they understand, things that can be visualised, things that they find interesting or weird. \citet{YenkimalekivanHeuven2013,YenkimalekivanHeuven2017} think that there are three tools that interpreters can combine to improve recall: imagination, association and location. The combination of these tools implies imagining a real location in which put and divide information, to visualise an image of the discourse listened to and to create associations between elements to help recall.

Another technique that \citet{YenkimalekivanHeuven2013,YenkimalekivanHeuven2017} suggest to improve long-term memory (LTM) is storytelling, that is reporting a story in the same language as the original heard without the help of notes. To recall a story, they highlighted the following techniques:

\begin{description}
\item[Categorization:] grouping together elements that share the same characteristics;
\item[Generalization:] drawing conclusions from examples or messages given in the text;
\item[Comparison:] pointing out differences and similarities among a series of elements, facts and events;
\item[Description:] describing the context in which there is an object, its shape or dimension.
\end{description}

\citet{YenkimalekivanHeuven2017} carried out a study about the effects of memory training on the quality of interpreting from Farsi into English with 24 consecutive interpreting students from the University of Applied Sciences of Teheran. For a semester they were divided into two groups: the control group received traditional training (listening and comprehension exercises in English) whereas for the study group part of the time was dedicated to imagination and storytelling exercises. Statistical analysis pointed out a positive effect of WM exercises carried out by the study group on the quality of interpreting, especially in reducing omissions.

The technique of generalization is also at the basis of an exercise to improve WM called parataxis (\citealt{BallesterJiminezHurtado1992}). In this exercise the trainer reads a list of elements that are linked (ex. eagles, hawks, kites, ospreys, buzzards etc.) and the students have to guess the general category (ex. birds of prey). The researchers also suggested the reverse exercise for LTM, which is called synonyms: it is a brainstorming activity in which, starting from a generic term, students have to recall as many examples related to it as possible.

Among the exercises to improve consecutive interpreting skills suggested by \citet{Gillies2013} there are paraphrasing, that is listening to speeches in a foreign language and repeating them in the same foreign language, and monolingual interpreting, which means reformulating a speech in your mother tongue using the same language. These exercises involve LTM and selective attention skills.

\citet{Gillies2013} also mentioned the LTM exercise of taking notes only after the speech has ended or the WM, LTM and selective attention exercise of taking notes during the speech but not using them during the translation. Instead, \citet{ChabasseDingerfelderStone2015} and \citet{SettonDawrant2016a} suggest doing consecutive exercises without notes. In this memory training activity students have to interpret very short texts, at the beginning (10 seconds), and then gradually interpret longer texts (up to two minutes) without taking notes.

As far as simultaneous interpreting is concerned, one of the most used preparatory exercises is sight translation, that is the oral translation of a written text, with or without previous reading (\citealt{Kalina1992,Kalina2000}, \citealt{PadillaBenitez2002}, \citealt{Gillies2013}, \citealt{SettonDawrant2016a}). This exercise involves selective and divided attention, it favours chunking of the discourse and anticipation skills.

A preparatory exercise for simultaneous interpreting about which there are divergent opinions is shadowing. This is the definition of shadowing given by \citet{Lambert1988}:

\begin{quote}
A paced, auditory tracking task which involves the immediate vocalisation of auditorily presented stimuli, i.e. word-for-word repetition in the same language, parrot-style, of a message presented through headphones. \citep[381]{Lambert1988}
\end{quote}

On the basis of the differences in the time between the moment when the interpreter receives the message and the moment when the message is translated (ear-voice span, \citealt{Goldman-Eisler1972}), \citet{AndresBodenFuchs2015} distinguished between three types of shadowing:

\begin{description}
\item[Phonemic shadowing:] the repetition of a sound just after having heard it;
\item[Adjusted lag shadowing:] \sloppy the student has to keep a certain distance (ex. 5--10 words) from the original text. \citet{PadillaBenitez2002} is in favour of this exercise;
\item[Phrase shadowing:] the student has to wait for the completion of an entire phrase before speaking.
\end{description}

\citet{Schewda_Nicholson1990} and \citet{TonelliRiccardi1995} mentioned also another variant of this exercise, that is multiple task shadowing: Shadowing is used as an exercise to divide attention since it has to be performed together with another activity, for example recalling the content after listening or answering comprehension questions. The same principle is at the basis of the exercise \emph{online cloze} (and error correction) (\citealt{Kalina1992}, \citealt{SettonDawrant2016b}) in which, while repeating the text, the student also has to fill in missing words identified by an acoustic signal or to correct mistakes.

Another exercise to train divided attention is Two questions at a time (\citealt{Kalina1992}, \citealt{Gillies2013}). In this exercise a person reads questions about a specific topic and another person has to answer. While the answer is being given, another question is asked. Questions and answers might be in the same language or in different languages. \citet{Gillies2013} described another variant of this exercise, in which the second person has to answer Yes/No questions and also to repeat them while they listen to the next question. Gillies thinks that this exercise is more similar to simultaneous interpreting because, differently from shadowing, it does not only involve speaking and listening at the same time, but also understanding.

On-line paraphrase (\citealt{RussoPippa2004}, see \sectref{sec:ghiselli:2.2}), also called smart shadowing in A or same-language simultaneous interpreting or within-language paraphrase (\citealt{SettonDawrant2016a}) is another useful exercise of divided attention to prepare for simultaneous interpreting.

Listening cloze is an exercise that many interpreters trainers suggested \citealt{vanDam1989}, \citealt{Kalina1992},  \citealt{PadillaBenitez2002}, \citealt{AndresBodenFuchs2015}, \citealt{SettonDawrant2016b}). The professor introduces a discourse mentioning speaker, topic and context. The speech is read and gaps, identified by an acoustic signal, have to be filled in. This a comprehension exercise where the student has to show to be able to grasp the speech meaning even if some information is missing.


\section{Data collection}


This section describes the data that were collected about autonomous exercise in the sample of interpreting students participating in this study. First of all, an overview of the characteristics of the sample will be given. Then the study plan of the master’s degree in interpreting will be briefly described. To conclude there will be a description of the monthly survey used to collect data.



\subsection{Participants}

In \tabref{tab:ghiselli:1} the characteristics of the students who took part in the monthly survey are summarised. Language A and Language B refer to the two languages of study, in which students took the entrance exam (see \sectref{sec:ghiselli:3.2}). Previous interpreting experience refers to any type of interpreting activity (liaison, consecutive, simultaneous or whispered interpreting) done during university courses, in a professional context or as volunteer work. 1stsubgroup and 2ndsubgroup refer to students recruited in two different academic years. Students from the first subgroup were recruited as volunteers in October 2015, when they were starting the first year of the master’s degree in interpreting. Data collection about autonomous exercise for them started in January 2016 and ended in February 2018. The second subgroup refers to interpreting students starting the master’s degree in October 2016. For this subgroup, data collection started in December 2016 and ended in February 2018. Mean age at T1 indicates the age at the first WM and selective attention test session. T1 varies from student to student because the test battery was done one person at a time by appointment. For the students in the first subgroup T1 is a day in November--December 2015, for the students in the second subgroup it is a day in October--November 2016.

\begin{table}
\begin{tabular}{lSSS}
\lsptoprule
& \multicolumn{3}{c}{Interpreting students (\textit{N} = 49)}\\
& {Subgroup 1} & {Subgroup 2} & {Total}\\\midrule
\textit{N} & 27 & 22 & 49\\
Mean age at T1 & 22.54 & 22.23 & 22.39\\
SD from mean age & 0.94 & 1.02 & 0.98\\
Male sex (\%) & 33.33 & 4.55 & 20.41\\
Right hand preference (\%) & 92.59 & 86.36 & 89.80\\
Left hand preference (\%) & 3.70 & 4.55 & 4.08\\
Ambidextrous hand preference (\%) & 3.70 & 9.09 & 6.12\\
Italian mother tongue (\%) & 88.89 & 100.00 & 93.88\\
French mother tongue (\%) & 7.41 & 0.00 & 4.08\\
Bilingual Ukrainian and Italian (\%) & 3.70 & 0.00 & 2.04\\
Language A English (\%) & 29.63 & 22.73 & 26.53\\
Language A French (\%) & 22.22 & 31.82 & 26.53\\
Language A Spanish (\%) & 7.41 & 13.64 & 10.20\\
Language A German (\%) & 14.81 & 18.18 & 16.33\\
Language A Russian (\%) & 18.52 & 13.64 & 16.33\\
Language A Italian (\%) & 7.41 & 0.00 & 4.08\\
Language B English (\%) & 40.74 & 63.64 & 51.02\\
Language B French (\%) & 18.52 & 13.64 & 16.33\\
Language B Spanish (\%) & 25.93 & 4.55 & 16.33\\
Language B German (\%) & 7.41 & 0.00 & 4.08\\
Language B Russian (\%) & 7.41 & 18.18 & 12.24\\
Previous interpreting experience (\% yes) & 62.96 & 45.45 & 55.10\\
\lspbottomrule
\end{tabular}
\caption{Study group characteristics\label{tab:ghiselli:1}}
\end{table}



\subsection{Study plan}
\label{sec:ghiselli:3.2}


The master’s degree in which students were enrolled during data collection was a two-year program that included the study of two foreign languages that students could choose among English, French, German, Spanish and Russian.

In order to enrol on the course students had to pass an entrance exam in both the languages of study and in Italian. The entrance exam was changed from academic year 2015--2016 to academic year 2016--2017. The first subgroup had to take cloze tests (filling in missing words) in Language A and Language B and recall tests (repetition without notes) of two speeches (about 4 minutes long): one speech in Language A, which they had to repeat in the same language, and one speech in Language B, which had to be repeated in Italian. They also had to do an on-line paraphrase exercise (see \sectref{sec:ghiselli:2.2}) in which they had to reformulate a text in Italian. The second subgroup took three cloze tests and three recall tests, one in Language A, one in Language B and one in Italian for each type of exercise.

Language A involved both interpreting exams (consecutive and simultaneous) from the foreign language into Italian and from Italian into the foreign language. Language B involved exams only from the foreign language into Italian, but students could add an optional exam of 6 ECTS from Italian into Language B. 12 students (4 from the first and 12 from the second subgroup) also took the optional exam. Overall, the compulsory interpreting exams counted for 26 ECTS for Language A and 22 ECTS for Language B. After passing all the exams, students also had to pass the final exams, that is a mock conference on a specific topic, during which they had to perform four interpreting tasks, two for Language A and two for Language B.



\subsection{Monthly survey}

Self-study is an important part of learning but it is difficult to measure because it implies the cooperation of the students, who have to give information about what they do outside the class. A balance was struck between getting reliable information and encouraging students to be constant over time in participating, by choosing to send them a monthly survey by email.

Data collection started the month after all the students had taken WM and selective attention tests the first time, that is in January 2016 for the first subgroup and December 2016 for the second subgroup. The questions were in Italian and aimed at knowing what the student had done in the day in which the email was sent. The emails were always sent in the evening and asked about that specific day.

This procedure had a double goal. On the one hand it aimed at helping students to recall what they did since it referred to exercises they had done little time before. On the other hand, the objective was favouring the reliability of the answers, since students were not asked to calculate how much time they devoted to exercise on average but they just had to think about a specific day. When students did not answer to the email, they were sent a reminder and encouraged to choose a different day or, if they preferred, to give data about an average day of that month. Students received the emails in random working days that changed every month.

The survey included the following questions:

\begin{enumerate}[noitemsep]
\item How much time did you devote to recall (hours/minutes)?
\item How much time did you devote to sight translation (hours/minutes)?
\item How much time did you devote to consecutive interpreting (hours/minutes)?
\item How much time did you devote to simultaneous interpreting (hours/minutes)?
\item Did you do another type of exercise (yes/no)?
\item (Only if you answered ``yes'' to 5) Which other type of exercise did you do?
\item Does the exercise you did today represent what you usually do in this period (yes/no)?
\item (Only if you answered ``no'' to 7): do you usually devote more or less time to exercise in a day (1: much less; 2: less; 3: the same; 4: more; 5: much more)?
\end{enumerate}



\section{Results}


This section aims at displaying the findings of the data collection. In 61.41\% of the answers participants declared that the day of data collection was representative of their exercise habits. When the day did not correspond to the average (38.59\%) in the majority of cases it was because students said they normally did more exercise. The average percentage of answers given over the emails sent was 85.2\%. Taking this into account, the data collected can be considered a representative sample of students’ study habits. The answers are based on participants’ own perception of their behaviour, which could be subjective, so this needs to be considered when looking into the results.

\tabref{tab:ghiselli:2} displays the data of both subgroups. The same data are shown from two different perspectives: duration and frequency of exercise. Duration is expressed with the mean of minutes devoted to exercise daily. Frequency is expressed with the percentage of exercise done out of the number of data collection requests, that is how frequently students did exercise, independently from the time they devoted to it.

Since this data collection was part of a broader PhD project, data were divided according to the dates of test sessions in order to be able to compare autonomous exercise with test results. The first subgroup took the test three times (November--December 2015, May--June 2016 and April--June 2017), whereas the second subgroup took the tests only twice (October--November 2016 and April--June 2017). For both subgroups T1 corresponds to the beginning of the first year of the master’s degree and T2 to the end of the first year. T3 for the first subgroup corresponds to the end of their second year of the master’s degree. Data collection about self-study stopped in February 2018 because it is when the last exam session of the previous academic year takes place. The type of data analysed were:


\begin{itemize}

\item ex: exercise, independently from the type of activity;
\item sim: simultaneous interpreting exercise;
\item cons: consecutive interpreting exercise;
\item sight\_tran: sight translation exercise;
\item rec: recall exercise.
\end{itemize}


For every data set, mean values with the corresponding standard deviations were calculated for the following periods:

\begin{itemize}
\item T1T2: the first year of the master’s degree;
\item T2T3\_or\_after\_T2 : the second year of the master’s degree. The only difference is that for the 1st subgroup there are more data since the monthly survey was sent during all the second year, whereas for the 2nd subgroup data collection stopped at the end of the first semester;
\item after\_T3 (1st subgroup only): between the end of the second year of the master’s degree and the final exams;
\item overall: all the data collection period, which is longer for the 1st subgroup (26 months instead of 15).
\end{itemize}


\begin{table}
\begin{tabular}{lS[parse-numbers=false,separate-uncertainty=true,uncertainty-separator={\,}]
                 S[parse-numbers=false,separate-uncertainty=true,uncertainty-separator={\,}]}
\lsptoprule
Type of exercise & {Duration} & {Frequency}\\\midrule
ex\_T1T2 & 63.74 (45.74) & 0.41 (0.2)\\
ex\_T2T3\_or\_after\_T2 & 94.62 (72.24) & 0.4 (0.17)\\
ex\_after\_T3  & 131.94 (75.41) & 0.42 (0.18)\\
ex\_overall & 88.43 (50.6) & 0.41 (0.16)\\
sim\_T1T2 & 21.94 (22.54) & 0.5 (0.31)\\
sim\_T2T3\_or\_after\_T2 & 42.18 (38.46) & 0.62 (0.26)\\
sim\_after\_T3 & 68.2 (43.84) & 0.73 (0.23)\\
sim\_overall & 37.86 (25.84) & 0.59 (0.22)\\
cons\_T1T2 & 27.57 (21.31) & 0.6 (0.3)\\
cons\_T2T3\_or\_after\_T2 & 38.67 (40.62) & 0.59 (0.26)\\
cons\_ after\_T3 & 49.73 (29.49) & 0.68 (0.25)\\
cons\_overall & 34.35 (24.98) & 0.6 (0.22)\\
sight\_tran\_T1T2 & 9.14 (9.22) & 0.36 (0.27)\\
sight\_tran\_T2T3\_or\_after\_T2 & 11.15 (11.72) & 0.38 (0.29)\\
sight\_tran\_after\_T3  & 10.5 (9.96) & 0.34 (0.29)\\
sight\_tran\_overall & 10.37 (7.95) & 0.37 (0.24)\\
rec\_T1T2 & 5.09 (10.29) & 0.18 (0.29)\\
rec\_T2T3\_or\_after\_T2 & 2.63 (5.41) & 0.14 (0.26)\\
rec\_ after\_T3 & 3.51 (12.41) & 0.11 (0.25)\\
rec\_overall & 3.31 (6.88) & 0.14 (0.23)\\
\lspbottomrule
\end{tabular}
\caption{Data about autonomous exercise. Column \emph{duration} lists the mean of daily minutes. Column \emph{frequency} lists the mean percentage of days. Values in brackets are standard deviations.\label{tab:ghiselli:2}}
\end{table}

The self-study profile that emerged is very diversified, with a high standard deviation, especially in the data about  recall,. the majority of students did not do this exercise. On average, data about frequency are more homogenous than data about duration.

Recall exercises were performed frequently at the beginning of the master’s degree and less and less as time passed. This is not surprising since recall exercises are considered as a preparatory activity for interpreting, consecutive interpreting in particular. Sight translation was done constantly over time, the same as consecutive interpreting. The exercise of simultaneous interpreting increased from the first to the second year. During the first year, simultaneous interpreting is gradually introduced in the lessons, so it is normal that this type of interpreting exercise was done more in the second year of training. Overall, the mean of minutes devoted to exercise was 63.74 (SD = 45.74) between T1 and T2, 94.62 (SD~= 72.24) between T2 and T3 or after T2. As far as the frequency is concerned, students said they did at least one type of exercise in 40\% of the answers. After T3, the duration of exercise for the 1st subgroup is much longer than before, 131.94 (SD = 75.41). This is probably due to the fact that in this period students had final exams, which are very stressful and demanding.

To conclude data description, the answers given to questions 6 are represented in \figref{fig:ghiselli:1}. Question 5 asked whether the student did another type of exercise and, if yes, question 6 was an open question asking which other type of exercise was done.

The mean percentage of affirmative answers to question 5 was 35.66\% and 85\% of the students declared at least once to have done a type of exercise different from those mentioned in the other questions. The answers given to question 6 were divided into 12 categories:


\begin{enumerate}[noitemsep]\label{list:01:categories}\sloppy
\item Shadowing (see \sectref{sec:ghiselli:2.3}): repetition of a text in a foreign language while listening. It was done to improve pronunciation, to learn useful expressions in the foreign language and to do exercise on fast speeches;
\item On-line paraphrase (see \sectref{sec:ghiselli:2.3}): rephrasing of a text in the same language as the original while listening;
\item Liaison interpreting: interpreting from and into a foreign language in turns to help two people or groups to communicate. Students mainly did this activity in trade fairs as internship;
\item Whispered interpreting: whispered translation of a speech for a small group of people, who is next to the interpreter. Students mainly did this activity as internship;
\item Terminology research and study;
\item Note-taking and symbol creation: students worked on the way they took notes during consecutive interpreting and tried to speed up this process by creating personal symbols to take note of recurrent concepts;
\item Listening in a foreign language;
\item Reading in a foreign language;
\item Written translation;
\item Transcription (of audio documents);
\item Self-correction: listening to your own interpretation to assess it, correct mistakes and think about better translation solutions;
\item Other: answers given only once or just by one participant.
\end{enumerate}



\begin{figure}
  \begin{tikzpicture}
    \begin{axis}[
      ybar,
      width=1\textwidth,
      height=.3\textheight,
      axis y line*=left,
      axis x line*=bottom,
      yticklabel=\pgfmathprintnumber{\tick}\,$\%$,
      ytick= {10,20,30,40},
      ymax=45,
      ymin= 0,
      xtick = data,
      xticklabel style={align=center},
      xticklabels = {1,...,12
%             {Shadow-\\ing},
%             {On-line\\para-\\phrase},
%             {Liaison\\interpre-\\ting},
%             {Whis-\\pered\\interpre-\\ting},
%             {Termi-\\nology\\research\\and study},
%             {Note-\\taking and\\symbol\\ creation},
%             {Listening\\in a\\ foreign\\ language},
%             {Reading\\ in a\\ foreign\\language},
%             {Written\\translation},
%             {Transcrip-\\tion},
%             {Self-\\correction},
%             {Other},
          },
          xlabel = {Categories (see list on page~\pageref{list:01:categories})},
          nodes near coords={\pgfmathprintnumber\pgfplotspointmeta\%},
      ]
      \addplot[draw=lsMidDarkBlue!80!black,fill=silpone] coordinates{(1,6)(2,1)(3,3)(4,2)(5,33)(6,4)(7,20)(8,13)(9,1)(10,1)(11,7)(12,9)};
    \end{axis}
  \end{tikzpicture}
\caption{Additional type of exercise}
\label{fig:ghiselli:1}
\end{figure}

% % %% Percentage Values:
% % % Shadowing 6, On-line paraphrase 1, Liaison interpreting 3, Whispered interpreting 2, Terminology research and study 33, Note-taking and symbol creation 4, Listening in a foreign language 20, Reading in a foreign language 13, written translation 1, Transcription 1, Self-correction 7, Other 9
\figref{fig:ghiselli:1} shows that the three more common typologies of exercises mentioned by students were terminology research and study (33\%), listening in a foreign language (20\%) and reading in a foreign language (13\%).


\section{Discussion}
As mentioned before (see \sectref{sec:ghiselli:2.2}) there are only two studies about autonomous exercise habits of interpreting students, \citet{Fan2012} and \citet{Wang2016}, so the present paper is one of the earliest contributions to this topic. Data are based on what students reported, so they might not be accurate. At the same time, the researcher was not their professor and they did not get any rewards for their participation in the study, which was on a voluntary basis, so there are no apparent reasons why they should have lied.

In the study of \citet{Fan2012} data about the autonomous exercise of interpreting students were collected over an academic year. In the first semester the mean daily time devoted to exercise was 90 minutes, whereas in the second semester it was of about 105 minutes. In this study, if all types of exercises are taken into account, the mean daily time devoted to self-study was 88.43 (SD = 50.6) minutes. Time devoted to exercise increased over time, like in Fan’s study. The length of self-study activities found by Fan was also confirmed, since in both studies the result is that students devote about one and a half hour a day to autonomous exercise. In this study, as the high standard deviation shows, the habits changed a lot from student to student.

Recall is a preparatory exercise and this study clearly shows that students devoted less and less time to it as they advanced in training and tended to focus only on consecutive and simultaneous interpreting. Recall can, however, be considered a valuable exercise also for a more advanced student. It requires a lot of concentration, a skill which is essential both in consecutive and in simultaneous interpreting. It implies to rely only on one’s mental resources without the help of notes. Notes are a valuable and necessary help in consecutive interpreting, but they cannot replace logic and critical thinking, so the interpreter can never be too dependent on them and has to make an effort to create a mind map of the message to avoid saying contradictions.

Terminology research, listening and reading in a foreign language were not included in the survey questions because the PhD project focused on cognitive aspects of interpreting. The fact that students mentioned these activities of language improvement as further self-study is in line with what trainers recommend. Reinforcing linguistic skills is in fact very important to achieve a good interpreting performance.

Students who did shadowing exercises said they did them in the foreign language to get used to fast speakers and to improve their accent and learn new expressions. This exercise is normally considered a preparatory exercise for simultaneous interpreting, but in this data set students declared to do this exercise in the first year of the master’s degree (8\% of the answers between T1 and T2), in the second year (4\% of the answers between T2 and T3 or after T2) and, in the data collected after T3 from the students of the 1st subgroup, shadowing was also mentioned after the end of the second year classes (10\% of the answers). Over the entire period, the percentage of shadowing was 6\%. It emerged that, differently from what could be thought, students considered shadowing a useful exercise not only in an early stage of training but also in an advanced stage.

Eventually, the percentage of times when students declared to have done any type of exercise (40\%) was quite low, since it means that in more than half of the days when they were asked, they did no exercise. This goes against what trainers would expect, but it might be due to long hours spent in class, which leave little time and energy for self-study.


\section{Conclusion}


The data displayed here are part of a wider PhD project for which other data were collected and comparisons between different data sets were done. The present paper deals with data collected about autonomous exercise and has a descriptive approach.

From the data collected it came out that self-study habits among students were very diversified, but the mean time devoted to interpreting autonomous exercise was in line with the findings of \citet{Fan2012}, that is about one and a half hour per day.

Exercise focused more on consecutive and simultaneous interpreting activities and less on support exercises such as sight translation and recall exercises.

In the open question, where students could mention other exercises they did, most of the time they mentioned terminology research and study and listening or reading in a foreign language. This is in line with expectations, since language study and vocabulary learning are life-long learning activities for an interpreter.

Shadowing was also mentioned among other type of exercises. Shadowing presents advantages and disadvantages \citep{Kurz1992} but if a more difficult version of this exercise is done, that is repeating a text in a foreign language having some difficult elements, such as a high speed rate or a difficult accent of the speaker, it may be useful to improve listening and speaking skills in a foreign language. Various scholars (\citealt{Kalina1992};  \citealt{PadillaBenitez2002}; \citealt{Gillies2013}; \citealt{SettonDawrant2016b}) suggested matching shadowing with other exercises, such as online cloze tests and comprehension questions afterwards to check whether also the message, and not only the words, was understood.

On-line paraphrase was mentioned by students as a further exercise, but only in 1\% of the answers. This exercise is more difficult than shadowing and implies a thorough understanding of the message. Another exercise, that was not mentioned by students and for which both concentration and understanding are necessary, is \emph{two questions a time} (\citealt{Kalina1992}; \citealt{Gillies2013}) (see \sectref{sec:ghiselli:2.3}).

Further developments could be carrying out an experimental study \citealt{YenkimalekivanHeuven2017} is an example) to see whether more support exercises such as recall, shadowing and on-line paraphrase, that some participants mentioned, or Two questions at a time, that none of the students did, would be useful for students to improve their interpreting skills. The potential improvement of interpreting skills through targeted exercises could be verified using real interpreting tasks instead of psychological tests. In this hypothetical experimental framework, a language level assessment would be necessary to see whether the language proficiency of students is comparable.

\sloppy\printbibliography[heading=subbibliography,notkeyword=this]
\end{document}

%%% Local Variables:
%%% mode: latex
%%% TeX-master: t
%%% End:

\documentclass[output=paper]{../langscibook}
\author{Gisela Mayr\affiliation{Free University of Bolzano}\orcid{}}
\title{Building bridges between languages: How students develop crosslinguistic awareness in multilingual learning settings}

\abstract{The present qualitative case study investigates the acquisition of increased crosslinguistic awareness in a multilingual learning setting, and the different development attributable to the linguistic backgrounds of emergent multilingual students. The study was carried out in a secondary school in South Tyrol, belonging to the German-speaking school system, where plurilingual task-based modules were inserted in regular language lessons. The languages involved were: German, Italian, English, French, Latin and Ladin. Students were involved in complex plurilingual problem solving processes during the elaboration of the language production. Thanks to this, crosslinguistic awareness could be trained and fostered, as plurilingual negotiating processes arose from the plurilingual and multimodal input provided by the teacher. }
\IfFileExists{../localcommands.tex}{
  \addbibresource{../localbibliography.bib}
  % add all extra packages you need to load to this file

\usepackage{tabularx,multicol}
\usepackage{url}
\urlstyle{same}

\usepackage{enumitem}

\usepackage{pifont}

\usepackage{listings}
\lstset{basicstyle=\ttfamily,tabsize=2,breaklines=true}

\usepackage{./langsci-optional}
\usepackage{./langsci-lgr}
\usepackage{./langsci-gb4e}

\usepackage{langsci-plots} 

\makeatletter
\let\pgfmathModX=\pgfmathMod@
\usepackage{pgfplots,pgfplotstable}%
\let\pgfmathMod@=\pgfmathModX
\makeatother

\usepackage{siunitx}
\sisetup{output-decimal-marker={.},detect-weight=true, detect-family=true, detect-all, input-symbols={\%}, free-standing-units,table-align-text-pre=false,group-digits=false,detect-inline-weight=math}
\DeclareSIUnit[number-unit-product={}]{\percent}{\%}
\makeatletter \def\new@fontshape{} \makeatother
\robustify\bfseries % For detect weight to work

\usepackage{todonotes}

  \newcommand*{\orcid}{}

\renewcommand{\lsChapterFooterSize}{\footnotesize}

\makeatletter
\let\thetitle\@title
\let\theauthor\@author
\makeatother

\newcommand{\togglepaper}[1][0]{
%   \bibliography{../localbibliography}
  \papernote{\scriptsize\normalfont
    \theauthor.
    \thetitle.
    To appear in:
    Jorge Pinto \& Nélia Alexandre (eds.),
    Multilingualism and third language acquisition: Learning and teaching trends.
    Berlin: Language Science Press. [preliminary page numbering]
    }
  \pagenumbering{roman}
  \setcounter{chapter}{#1}
  \addtocounter{chapter}{-1}
}


 
  %% hyphenation points for line breaks
%% Normally, automatic hyphenation in LaTeX is very good
%% If a word is mis-hyphenated, add it to this file
%%
%% add information to TeX file before \begin{document} with:
%% %% hyphenation points for line breaks
%% Normally, automatic hyphenation in LaTeX is very good
%% If a word is mis-hyphenated, add it to this file
%%
%% add information to TeX file before \begin{document} with:
%% %% hyphenation points for line breaks
%% Normally, automatic hyphenation in LaTeX is very good
%% If a word is mis-hyphenated, add it to this file
%%
%% add information to TeX file before \begin{document} with:
%% \include{localhyphenation}
\hyphenation{
affri-ca-te
affri-ca-tes
au-ton-o-mous
Cha-basse
Din-ger-fel-der
plu-ri-lin-gual
Ya-na-pra-sart
Mi-ri-ci
Ström-quist
}

\hyphenation{
affri-ca-te
affri-ca-tes
au-ton-o-mous
Cha-basse
Din-ger-fel-der
plu-ri-lin-gual
Ya-na-pra-sart
Mi-ri-ci
Ström-quist
}

\hyphenation{
affri-ca-te
affri-ca-tes
au-ton-o-mous
Cha-basse
Din-ger-fel-der
plu-ri-lin-gual
Ya-na-pra-sart
Mi-ri-ci
Ström-quist
}
 
  \togglepaper[1]%%chapternumber
}{}

\begin{document}
\maketitle 
\shorttitlerunninghead{Building bridges between languages}%%use this for an abridged title in the page headers




\section{Introduction}

\subsection{Crosslinguistic awareness: A language learning competence}

Crosslinguistic awareness, according to Jessner, is the ability of multilinguals to make implicit or explicit use of the connections and overlappings that exist between the different language systems in the human brain during language production and use \citep[116]{Jessner2006}. It is a conglomerate of competences in the performance domain, including, on the one hand, the selective analysis of linguistic structures accompanied by a repertoire of abilities that allow the speaker to successfully handle deficiencies in language use and the ability to anticipate arising problems in communication by identifying and selecting appropriate strategies to overcome them. On the other hand, this presupposes a more general monitoring of the language-processing process and production, aimed at error analysis and correction as well as the optimization of communication (\citealt{LuoEtAl2010}; \citealt{DeAngelisDewaele2011}; \citealt{HerdinaJessner2002b}). It also implies the capacity of multilinguals to apply metalinguistic abilities to a certain context by activating specific linguistic resources across their languages. 

Recognizing, and availing oneself of crosslinguistic interaction (CLIN), entails the ability to make use of transfer and inference as well as code-switching, code-mixing, translanguaging, and crosslinguistic borrowing in oral communication. This means all interlingual correspondences, regularities and contrasts are employed to optimize communication. Multilinguals enact multilingual compensatory strategies \citep[87]{Jessner2006}, and are therefore able to switch to a metamode, where language production is constantly surveyed. (ibid. 87). The resulting metacognitive translingual transfer is, to a large extent, unconscious (\citealt{DeAngelisDewaele2011}; \citealt{GibsonHufeisen2011}; \citealt{Vidgren2013}), but if made conscious can become a competency, and a strategy to master complex multilingual communicative situations. It is also assumed that multilingual learners, in contrast to monolingual learners, not only rely on L1 for transfer, but in many cases prefer L2. The reason is that L2, unlike L1, is a consciously acquired language and can more easily provide comparable structures and words (\citealt[79]{HerdinaJessner2002b}; \citealt[64]{House2004}; \citealt[178f]{Muller-Lance2006}). This so-called L2-factor leads to the assumption that the L3-acquisition process differs substantially from the L2-acquisition process (\citealt{Hufeisen2011}; \citealt{Cenoz2013Influence}; \citealt{HerdinaJessner2002b}). Comparative studies have shown that early bilingualism has a positive influence on any further language acquisition (\citealt{CenozValencia1994}; \citealt{Lasagabaster1997}; \citealt{Pilar_Safont2003}; \citealt{Ringbom1987}). It is demonstrated that test- persons achieve higher proficiency in L3 English if they have achieved a high proficiency level in L1 as well as L2 \citep{DeAngelisJessner2012}. The latter study also shows that the L2 factor, and its influence on subsequent language acquisition is, to a large degree, dependent upon proficiency and the psychotypological perception of L2. If L2 is perceived as distant and unfamiliar, this will inhibit transfer from L2 to subsequent languages. 

However, in the ideal case, crosslinguistic awareness can lead to increased metalinguistic awareness, which in turn allows for cross-lingual lexical consultation. In this case, procedural knowledge, the knowledge about how something is done, as well as declarative knowledge, the basic knowledge about something, is drawn mainly on L2, and L1 loses its predominant role for transfer. As a result, grammatical error recognition and analysis become much more effective and productive \citep{Bialystok2004,GibsonHufeisen2011}. This induces multilinguals to be more risk-taking during language production, since they can avail themselves of increased cognitive control. Cenoz postulates the existence of different levels of metalinguistic awareness, which have an effect on the plurilingual lexicon and support multilingual speakers \citep{Cenoz2013Influence}. They can rely on a metasystem: the interlanguage formed during L3 acquisition \citep[131--161]{HerdinaJessner2002b}. 

All these competences, however, are not to be considered as given, but rather develop when language systems interact with each other. Therefore, crosslinguistic awareness can be exercised by adopting specific plurilingual learning settings that promote multilingual strategies in the classroom. Thanks to this, the consciousness of the different interrelations existing between diverse language systems could be trained also in monolinguals, and their ability to draw on implicit and explicit declarative as well as procedural knowledge to determine similarities, and differences raised. This initiates a process in which all languages may take on the role of bridge languages, and assume different functions, according to the specific needs of the speaker. Due to their increased ability to handle multilingual discourse, multilingual speakers in such a context often assume model role function in conversation and positively influence the communicative competence, and language learning process of less proficient students. Thus, multilinguals can practice their abilities in a learning setting, where there normally is no space for transfer and CLIN, and at the same time take over the role of mediators between languages, cultures, and worldviews. This way, they can initiate individual learning processes that best comply with their multilingual biography, and in the meantime enhance the learning process of monolingual learners. \citep{Jessner2006}.

Relatedly, crosslinguistic awareness is not only relevant on the linguistic level, but can initiate transcultural learning as well, by acknowledging that language acquisition processes strongly depend on the emotional dispositions of the learners towards the individual languages and cultures \citep[29]{Burwitz-Melzer2012}. This implies that languages are socially and politically charged, and their development associated with the historical and cultural development of a certain community at a given time. Crosslinguistic awareness, then, as intended in its political and social dimension, implies the critical questioning of power structures behind common language use by different subjects, social classes and cultures, as well as the consciousness of the presence of gender issues in language use (ibid. 29; \citealt[28f]{Morkotter2005}). This ultimately leads to the acceptance of differences in all their forms, and fosters an inclusive attitude with regard to the different languages of instruction in school, and in the society in general. 

In addition to these cognitive and socio-cultural aspects of crosslinguistic\linebreak awareness, the present study wants to analyze the effect of plurilingual interaction in linguistically heterogeneous groups, with regard to the possible change of attitudes and dispositions of the individual speakers. This is only made possible by reflecting critically upon each person’s language-learning history in comparison with that of others. In particular, monolingual learners can recognize attitudes, and dispositions of multilingual learners, and adopt them, with the result that these monolinguals can then more actively and autonomously participate in the multilingual problem-solving processes. 

In order to implement the above-mentioned forms of learning, plurilingual learning settings are required, which put a focus on the identity of the learner and their language biography. It is necessary to take account of the order, and mode of acquisition of each individual language, which altogether forms a complex system of languages and emotions. This means that each newly acquired language gives access to new experience and a new perception of the world. The acquisition of a second or third language is associated by the awareness that each person’s own identity is not unalterable but can change and expand, and that this expansion grows in complexity with each further language (\citealt{ReichKrumm2013}: 88). Such forms of crosslinguistic awareness imply the consciousness that each language biography is related to attitudes and emotions, and that these influence the way we communicate and perceive the world.

\subsection{Plurilingual task-based learning}

The tradition of task-based language learning (TBLL) can be seen as a further development of the communicative approach to foreign-language teaching (FLT), and has established itself as one of the most successful innovations in FLT over the last few decades. In task-based language teaching (TBLT), learners face meaningful and relevant tasks, and use the target language to solve real-world problems in a functional way. It is no longer the aim of teaching to impart grammatical structures, which in TBLT are acquired indirectly while students deal with diverse contents and tasks that involve them emotionally and cognitively \citep{Hallet2012}. Learners cooperate autonomously in groups, this way initiating problem-solving processes, where solutions are found in an act of collaborative learning, and where meaning is continuously re-negotiated among peers. Thus, a change in perspective takes place, and learning is no longer seen as the transfer of knowledge mainly provided by the teacher, but rather as the transformation of knowledge, performed by the learners themselves in autonomy \citep{Ellis2003}. This form of learning raises intrinsic motivation, and in the meantime focuses on the acquisition of learning strategies. It leads to the forms of self-reflection, which give the learner the opportunity to see themselves at the center of action, and make autonomous decisions with regard to their personal learning process. Using foreign languages as a realistic means of communication in the classroom gives the learner the impression of being a competent speaker by boosting self-confidence. This way the ideal conditions are created for the development of the communicative competence along with the forms of social learning (\citealt[84f]{Dewaele2010}). 

However, until now there have been no attempts to adapt TBLT to the needs of plurilingual teaching and learning at school, although various researchers have explicitly requested a change in perspective from monolingual to plurilingual forms of learning (\citealt{Kramsch2009}; \citealt{Hallet2015}; \citealt{MartinezSchroder-Sura2003}). The present research study is a first attempt to link TBLT to forms of plurilingual learning, thus fostering the ability of the individual to enact cultural and linguistic inclusion in a society characterized by pluralistic discourse. Plurilingual TBLT is an approach, where the input for the students is not only multimodal but also provided in more languages (5 languages in this case: German, Italian, English, French, Latin). 

While solving the task, students work with documents in more languages about the same or similar topics. They compare and analyze these documents in the course of the task solving process, complete the task and finally elaborate a plurilingual output (\citealt{Mayr2020}). This promotes a form of learning, in which the learners develop the ability to communicate not only in different languages simultaneously, but also to mediate between these languages, and cultural reference systems in a process of continuous comparison (\citealt[88]{MeisnerEtAl2009}). Being provided with plurilingual and multimodal input, the learners are given the opportunity to work with more than one language, and adapt their language use to different communicative needs or purposes. In this process of continuous mediation, and translation from one language to the other, the learners can develop and activate their multilingual repertoire, while in the meantime also increasing their crosslinguistic awareness. Thanks to the plurilingual TBLT-approach, it is therefore possible to promote crosslinguistic awareness, which also according to \citet{Allgauer-HacklJessner2013} can be acquired in a process of learning, under the condition that all languages are incorporated, and critical reflection is promoted. This is also the main objective of the multilingual task-based learning project introduced in this study.

\section{Research project and questions}

This case study was carried out at a secondary school in Bolzano, South Tyrol. The region is situated in Italy, and is characterized by a minority language situation, as a minority population of German speaking people live there. This population has been granted self-determination to a large degree, and can avail itself of an autonomous German-speaking school system, where Italian is L2 and English is taught as L3. However, due to a troubled history of coexistence between the two linguistic groups, there is still a negative influence on the psychotypological perception of Italian L2 as will be shown in the data analysis. Due to the L2-factor and its importance for language learning, this affects subsequent language acquisition. The students in the class subject of the case study, follow a particular language curriculum described below:

\begin{itemize}[noitemsep]
\item German as a medium of instruction
\item Italian L2 since the first year of primary school (four hrs. per week)
\item English L3 since the 4th year of primary school (three hrs. per week in primary and four hrs. per week in secondary school)
\item French L4 since the first year of secondary school (four hrs. per week), so only two years at the time of the beginning of the project. 
\item Latin since the first year of secondary education (three hrs. per week)
\end{itemize}

The students, thus, on average should have at least a B2 level (CEFR) in Italian, although in many cases the actual proficiency level is lower, a B1--B2 level in English and an A2 level in French. 

For the data collection, a group of four students was observed during plurilingual TBLT classes over a period of eight months, when five plurilingual modules of the duration of 10 hours each were inserted in regular language classes.  

Before the beginning of the project, all 22 students were administered a questionnaire on their language biography, and studies. In order to comply with the principle of the maximum possible diversification, and to be able to observe how the learning process develops under different conditions, it was necessary to find students with divergent social, cultural and linguistic backgrounds, as well as the proficiency levels and character. Four students were chosen on the basis of the outcomes of the questionnaires. The following criteria were adopted:

\begin{itemize}[noitemsep]
\item Linguistic background: this should provide a large range of possibilities, from monolingual to multilingual language biographies
\item Perception of languages: attitudes with regard to the single languages
\item Degree of proficiency in the different languages of schooling 
\item Frequency of use inside and outside of school 
\item Presence of heritage languages
\end{itemize}
Additional criteria:

\begin{itemize}[noitemsep]
\item Social behavior in the group
\item Regularity in attendance
\end{itemize}

During the implementation phase, three stimulated recalls (SR) were carried out with each of the four students at a two-month interval. In addition, audio and video recordings of the peer interaction among students during the negotiation processes, and the outputs were made and analyzed. The discourse analytical approach adopted was the “documentary method” (\citealt{BohnsackEtAl2013}). At the end of the project, a retrospective interview was carried out with each student. The data analysis focused on the following research questions:  

\begin{enumerate}
\item Does plurilingual learning increase crosslinguistic awareness?
\item What aspects of crosslinguistic awareness are improved in plurilingual TBLT?
\item How does the acquired crosslinguistic awareness affect further language learning?
\end{enumerate}

In the following outline of the data analysis, the statements of the students regarding crosslinguistic awareness, and the final interview are summarized, and integrated by the analysis of the audio and video recordings of the discursive processes. The students were anonymized, and given fictitious names.

\section{Data analysis}

\subsection{Student 1: Amelie}

\subsubsection{Background}

This student’s social background is characterized by diglossia, which means that she is used to switching from a standard German variety to the South Tyrolean dialect, commonly used in lifeworld discourse but also at school. This implies that she has the ability to switch from one variety of German to the other according to specific communicative needs. Amelie speaks the dialect at home, with her friends, and at school with her peers. A standard variety of German is usually used within the classroom, both during the lessons themselves, and when talking to the teachers. Amelie comes from a rural region, and therefore rarely has the occasion to speak Italian (the second most widely spoken language in South Tyrol). She has learned Italian only at school. Even though she has been learning Italian since her first year in primary school, she still considers her proficiency as only sufficient to get by, and perceives the language as distant and unfamiliar.


\subsubsection{The learning processes}

The student claimed that it was very difficult for her to learn to switch from one language to the other. At the beginning of the study, the language activation mechanism was still in a monolingual mode \citep{Grosjean2007}. In fact, when claiming that “when a text was in German it was not possible for me to speak about it in English for example”, (translation from German by the author), she referred explicitly to this difficulty. Only during the project did Amelie shift from a monolingual mode to a multilingual one. Thus, later she said that “the more you speak the better you get used to switching ... and if you don’t how to continue, another language may be helpful”. Amelie here found that the languages in her repertoire complemented and supported each other. This also implied that during the language production process, code-switching assumed a scaffolding function in order to help her cope with difficult situations. She also realized that code-switching could be used in a multilingual discourse for different strategic purposes.

She stated that she had learnt to use previous knowledge and experience as well as new crosslinguistic knowledge to help accelerate her learning process in all languages. Amelie learnt that it was helpful to expose oneself to challenging, multilingual situations, and develop compensatory strategies to handle them. 

The data show that the student activated self-regulatory forms of learning, which meant that she activated strategies and metacognition to identify areas of learning, and sought out the most suitable strategies to enhance her learning process. As a result of this, the student used predominantly French in the negotiation processes at the beginning of the project. When she said that “the French pronunciation is the most difficult one, and thanks to the practice during these modules it became better and better” she demonstrated that she chose a language for practice she didn’t often have the opportunity to speak. Her language choice and progressive improvement showed that multilingual learning increased the student’s consciousness about her different levels of proficiency in the different languages of her repertoire. Thanks to the learning setting, she was capable of taking action and found new ways of learning. Amelie learnt to reflect on her own language production and to observe it from an outside perspective so as to critically analyze and correct it wherever necessary (in fact she continuously looked up the correct pronunciation for French words). Furthermore, she understood that she could draw on her functional multilingualism to handle demanding situations and mastered also sub-areas of technical languages.  

The audio recordings show that, during the learning process, she would first assess her interlocutors’ language proficiency, and then adapt her language use to their needs. This allowed her to reflect on her own attitudes and habitual language use. She showed that she had acquired an inclusive attitude towards students who were new in the class and didn’t speak any Italian when she said that “always when we spoke with him we avoided Italian, because we didn’t know whether he understood us”. 

Amelie herself had quite a negative psychotypological perception of Italian before the start of the project, as the findings of her questionnaire showed. As a result, her motivation to activate and use Italian only manifested itself in the last module. When she said “I started switching to Italian too, which I didn’t do at all at the beginning”, she signaled a turning point in her language-learning process, because, as she began to include Italian L2 in her active multilingual repertoire. The way she perceived the language had changed, and consequently the further language acquisition process had, too. The multilingual learning setting allowed forms of social learning and imitation that gave the student the opportunity to change her perception of Italian, thereby rendering it accessible to her as a source of transfer and CLIN for L3/Lx. The disposition of the student changed and she opened up to plurilingualism. This allowed her to resort more and more to crosslinguistic lexical consultation and to understand that there was occasionally no one-to-one correspondence in the meaning of words in different languages, and that sometimes it was simply not possible to translate a word from one language to another. She stated that especially when analyzing literary texts “when you switch from one language to the other, and you want to use the same word, you realize that you can’t translate, but you need to find an appropriate word.” In this context while she was trying to find an appropriate translation for the dialect word \emph{tratzen} (the dialect word for tease) and she wasn’t able to, Amelie also recognized that the South Tyrolean dialect was her language of emotional socialization \citep{Pavlenko2011}, and that it allowed her to best express particular emotions. Plurilingual literary learning helped the student to tackl the problem of polysemy and ambiguity in the plurilingual discourse, while also developing a sense of transcultural awareness that contributed to her overall awareness of the plurality and heterogeneity as well as hybridity of cultures \citep[201]{Hufeisen2010}. She also learned to influence the course of the conversation by strategically code-switching when she wanted to achieve a certain effect. For example, when her peers were losing concentration and started using the German dialect, she brought them back on track by switching repeatedly to Italian using the expression \emph{ehm iniziamo?} (`can we start?').

\subsection{Student 2: Sarah}
\subsubsection{Background}
\largerpage
This student comes from the Gardena valley, where Ladin is still spoken, mainly as a heritage language. According to the assertions in her questionnaire, Sarah speaks Ladin and South Tyrolean dialect in her family, and also claims that Italian is spoken in a wider familiar context with relatives. She states that within her family she assumes different language roles, depending on her interlocutor. She never uses Ladin at school or in class. It is an unspoken rule for her to restrict the use of this language only to her family, and to the sphere of her strictly Ladin- speaking friendships. Because of this she also feels that part of her personality is excluded from her scholastic career.

\subsubsection{The learning processes}

Sarah was a very active student and it could be observed that during the study she developed the ability to observe the conversation from a meta perspective, and monitored it. Sarah then intervened with regulating strategies when she realized that the conversation drifted off-topic and her peers were losing concentration, by repeatedly switching to Italian and using the imperative \emph{concentriamoci} (`let’s concentrate'). Sarah had become the reference person for all the questions and doubts concerning Italian, since this was the language she knew better than anyone in the group. She became a role model and used the deriving authority repeatedly to bring her peers back to concentration. Through her language behavior in the group, she uncovered linguistic hierarchies. Thanks to her repeated switches to Italian and French she induced her peers to reduce the use of English, the predominant language in this group, and use other languages more frequently. Sarah boosted the use of Italian, which was the language least often chosen at the beginning of the project. Sarah like Amelie often resorted to the languages she wanted to practice, and this way learnt to judge her linguistic knowledge in the different languages using translingual criteria. She compared her linguistic competences in the different languages, and made good use of the gained knowledge for her further learning process. When translating from one language into another, she resorted to her complete multilingual repertoire as a source. She included all the languages of instruction in her learning process, so that in a continuous activity of mediation these could support and complete each other. This way language monitoring was activated, and supported her language acquisition. In the interview she stated: “now when I read a book in a foreign language, and I don’t understand something, I simply deduce it from other languages, instead of using a dictionary. I even use Latin, it helps to be able to switch from one language to another, to solve problems”.

Sarah learned to use previously acquired knowledge, strategies, linguistic and content-specific know-how, as well as procedural knowledge on different levels to cope with difficult situations, and to speed up her language-learning process. This student too used codeswitching as a scaffolding strategy in linguistically demanding situations during language production. This required her to evaluate her own linguistic abilities in the different languages correctly as well as those of her interlocutors, in order to monitor her language production, and intervene with corrections wherever necessary. Meaning was identified through a process of transcultural and crosslinguistic analysis and comparison. Sarah perceived her multilingual repertoire as a network of interrelations and intersections, which formed a whole, and continuously edited her language production. The statement “the project helped me to find different ways to express myself without having to switch to German all the time” implies that  during the learning process, a generic multilingual repertoire could develop that made available different patterns and strategies to acquire the knowledge of the world, and thus form a critical personal opinion. Sarah gained more and more the ability to deduce unknown words from her knowledge of other languages. This meant that her ability to transfer knowledge and strategies from one language to the other had accelerated, which implied increased levels of language monitoring and crosslinguistic awareness. 

Her heritage language, Ladin, however, remained excluded from the language-learning process and found no place within the school. But she became conscious of this, and the fact that her language use depended on the situation and the social setting. Sarah acknowledged that she often used Ladin for transfer, and that it was always present but never actually used at school. Multilingual literacy training helped her realize that different literary texts belonged to different cultural reference systems and that, thanks to these plurilingual modules, different discourse worlds and genres were brought into contact with each other. Sarah learnt that different languages were associated with different emotions for her, and that these emotions had a biographical origin. For instance when using Italian she says that “it is the language I associate with my father, the memories I have of him speaking in front of an audience”.

\subsection{Student 3: Vera}
\subsubsection{Background}
This student lives in an urban area and has a bilingual German/Italian background. However, the student does not define herself as bilingual in the questionnaire, and her assertions about the two languages are contradictory. Vera gives German a prominent position, even though her biography is clearly bilingual with a predominance of Italian in certain spheres. Her attitude reflects a different emotional perception of the two languages as well as a lack of awareness about the actual nature of her multilingual background.

\subsubsection{The learning processes}

Vera was a very active and interested student, and in contrast to the other three students, already felt at ease in multilingual situations at an early stage of the project. This allowed her to become a role model for the other students in the group. She often employed codeswitching and translanguaging as discourse-stra\-te\-gic instruments, which was an indicator that her multilingual communicative competences were quite highly developed right from the beginning. 

There were many examples where the student showed that CLIN was a means for her to monitor her spoken production and correct it when necessary. Right from the beginning she would make decisions about grammatical and lexical correctness, using crosslinguistic consultations in different areas. This demonstrated a high level of crosslinguistic awareness and an efficient language monitor. There were several examples in the recordings that showed that her multilingual repertoire was constantly present and that translanguaging was naturally part of her and her language production.  She could apply her language monitor according to the specific linguistic needs that emerged during the communication. In the sentence “he who dares der wagt to crave means to want something” for example she uses translanguaging as well as paraphrasing in one sentence to explain a verse from a poem to her peers. 

In the process of language mediation, Vera realized that it was not always possible to translate words from one language to another, and that meaning and connotation varied according to the historical development of the different languages. When she states that “when dealing with Christmas in different cultures, it becomes evident that it has many different meanings” she shows that language mediation, for her, meant playing with different meanings in the light of different cultural reference systems. This ability also allowed her to create new meaning by combining various connotative aspects, so for instance she created the term “traffic decoration” for “Christmas decorations,” combining the German “Straßendekorationen” with the English word. 

Vera was, to a large extent, indifferent towards language hierarchies. The multilingual setting helped her live out her plurilingual identity, which supported her in further language acquisition.  The multilingual learning context provided greater clarity for her with regard to her own multilingual background. She could experience herself consciously as a multilingual subject, and identify herself as such in front of the others. In the meantime, she realized that the process of multilingual language production accelerated with time and that this reflected the degree of activation of her multilingual repertoire, she claimed that all the languages of her repertoire “seem as if they were one language only”. On a semantic level, the student discovered that polysemy was linked to multilingualism and that meaning was often so multifaceted that it reflected reality “like a kaleidoscope”. 

Transfer, in this case, occurred on many different levels such as: grammatical, lexical, textual but also social and emotional level. On the lexical level, for instance, there was evidence that instead of just single words or expressions, Vera developed the ability to transfer chunks that could be combined and recombined within one language or between languages, according to her needs. This induced her to use multilingual metaphors allowing her to express meaning by combining and overlapping different metaphorical and semantic fields, thereby reaching a high degree of expressive complexity. The student could infer unknown meaning in forms of intercomprehension-learning by using her multilingual repertoire (\citealt{HufeisenNeuner2004}; \citealt{MeisnerSchocker2005}; \citealt{MeisnerEtAl2009}). According to her, multilingual learning allowed her to create new bridges between languages and these bridges helped her to reach \emph{das Ganze} (`the whole'). A statement that showed increased awareness about the fact that the multilingual repertoire is a transient and ever-changing unity, made up of different languages intricately linked to each other. Thanks to working with multiple meanings, Vera could develop a refined perception of slight semantic nuances provided by different languages. 

\largerpage
This student, like the two previous ones, developed the ability to better judge her interlocutors’ linguistic proficiency and dispositions and consequently regulated her behavior so as to allow successful communication. Vera could identify herself as a multilingual speaker in front of the others, and use her ability to mediate between the different worlds at their disposal. This not only changed her but also the others’ attitude towards the languages used in the course of the project. English was no longer the dominant foreign language, as was commonly the case due to its social prestige. Instead, other languages were perceived as equally important and interesting, and therefore became more familiar. 

\subsection{Student 4: Andrea}
\subsubsection{Background}
Andrea grew up in a German-speaking family in an urban area. Within her family, she speaks exclusively the local German dialect. Her statements are in many cases contradictory but they show that she has a rather negative view of her language competencies in the other languages. Although she lives in an urban area, where German as well as Italian are commonly used, she claims that she almost never has the opportunity to use Italian in her free time, and that her acquisition of the language has been restricted mainly to school, and the interactions with the teacher in class. Andrea therefore perceives her learning of Italian to have plateaued.

\subsubsection{The learning processes}

The student claimed that at the beginning of the project it was very challenging for her to switch between languages and that she always had to think about how to proceed. Over the course of the project, however, thanks to a habituation process, the use of multiple languages and codeswitching became easier and easier for her. Andrea stated that the activation of her multilingual repertoire did not only imply a quantitative enhancement in her spoken production, but also a qualitative enhancement because she learnt to use different languages as a communicative strategy, which meant that she used codeswitching to express specific meaning. Thanks to these strategies she was able to acquire new words in more languages, in fact she asserted that “I am able to express myself better, because I have learnt more complex words”

Over time, thanks to the forms of social learning, the students became aware of the fact that the error correction could take place in autonomy, and in a crosslinguistic mode, using more than one language to facilitate the process. Her statement that “sometimes if you translate a word you need to find a synonym, or to paraphrase it, or make a description, but this is not important the important thing is to make yourself understood”, showed that she not only made use of compensatory strategies when needed, but thanks to the forms of imitation learning, she stored knowledge that she could then activate at a later point in time. Eventually, she tried to use Italian more and more, but her language production was characterized by frequent mistakes, which in most cases she was not able to correct. However, Andrea became aware of the fact that her Italian repertoire lacked colloquial forms, due to the exclusive use of this language in class. The student therefore sought to increase her competency in this field as well, by using Italian in colloquial situations. This meant that, thanks to the multilingual learning setting, this student too acquired a more detailed awareness of her competences in the different languages, and used forms of self-regulated learning to adjust to her evolving needs. Andrea, like the other students, applied her functional multilingualism to enhance her communicative effectiveness, and was thereby able to adjust her communicative behavior to the requirements of her interlocutors. 

Her psychotypological perception of Italian L2 changed towards the end of the project, since she perceived herself as a more competent speaker of this language, in fact she asserted that working with different languages at the same time “reduces prejudices”. This allowed Andrea to make better use of L2 for transfer, which in turn gave her the possibility to play her part, when she was the only one in the group to find the Italian translation for \textit{scar} (\textit{ciccatrice}). At the same time the simultaneous use of more languages helped Andrea to overcome preconceptions, and to open up new access points to the different languages that “are all of the same importance”. Thanks to her newly acquired plurilingual reading skills, she was able to transfer knowledge from one text to the other, and thereby realized that the construction of meaning in a reading process based on plurilingual inputs was much more complex than in monolingual reading. Contrastive multilingual reading exercises enhanced her critical view of historical and social phenomena, and she realized that the construction of meaning in the reading process in many cases relied on complex crosslinguistic and transcultural word knowledge. So for instance she realizes that the word \emph{patriotic}\slash\emph{patriotisch}\slash\emph{patriotico}\slash\emph{patriotique} had profoundly different connotations in the different languages, and that this fact was attributable historical reasons. Andrea also acquired items of subject-specific vocabulary in in this context, and used them adequately. 

At the end of the project, the student claimed that she had begun to perceive all the different languages as one single language, and that in addition to critically analyzing different languages and cultures, multilingualism in itself had become a new culture to her, and a new way of being in the world.

\section{Conclusions}


To pick up on key aspects it can be said that plurilingual TBLT initiates different learning processes in the learners, depending on their language biography. This creates a plurilingual learning setting that helps learners activate their plurilingual resources, and make use of them in communication. This way, their linguistic repertoire is expanded, and both receptive as well as productive skills in all languages are developed. This form of learning forces the learner to constantly compare both linguistic and cultural content, and consequently to develop and expanded awareness of the similarities and differences between languages and cultures. Students undergo a consciousness-raising process, that leads them to better judge their own linguistic competences in the different languages. They learn to use their multilingual repertoire as well as the interrelated resources at their disposal to optimize their language production. Therefore, with regard to the first research question, it can be stated that plurilingual learning with TBLT methodology increases crosslinguistic awareness in students, and that the degree and the extent to which it is increased is strongly dependent upon biographical aspects. All students in fact developed an increased crosslinguistic awareness, but the degree to which this awareness developed was strongly influenced by their linguistic background and language biography as well as prior knowledge. 

With regard to the second and third research questions, the data clearly show that there is a focus on the activation of transfer strategies, which are adopted at different levels to overcome linguistically demanding situations. Most notably, lexical transfer and cross-lexical consultation between L2 and L1 can be mentioned here. While plurilingual students already have access to transfer strategies, thanks to the practice in the classroom, they become aware of how these strategies can be used for communication as well as for their own language-learning process. This leads to the ability to activate the use of transfer for every language in their repertoire, where the choice is determined not so much on how long a language has been learnt, or the language family, but rather by the needs, and whether in a specific communicative situation one language can be more useful than another to provide solutions to a certain problem. Students, who come from a predominant monolingual social background, and have experienced mainly consecutive language learning on, can approach transfer from L2 and L3 thanks to imitation learning and forms of social learning, and this way also change both their psychotypological perception and attitude towards their L2 (Italian). The increased attention, and the simultaneous use of more languages enables all the learners to activate the forms of intercomprehensive learning by making use of their linguistic and translingual knowledge to deduce the meaning of unknown words, and expressions. All students claimed that multilingual learning accelerated their language-learning process, as they experienced new ways of language acquisition. 

At the same time, students also tend to transfer meaning from one language to the other, thus changing the composition of their own system of cultural and linguistic reference. Thanks to contrastive reading exercises, they learn that the construction of meaning is culture-specific, and that multilingual discourse is characterized by ambiguity, fluidity and polysemy. Due to this, the meaning-making process becomes transcultural and more complex, since it is based on more than one cultural reference system. This gives the learners the opportunity to play with multiple meanings and metaphors by recomposing and repositioning them in new and unforeseen ways. 

Communicative strategies such as codeswitching and translanguaging on the one hand are implemented by students with a less plurilingual background (students 2 and 4) to overcome difficult linguistic situations, thus allowing them to regulate their learning process in a way that helps them identify problem areas, and search for new ways of learning. On the other hand, students with a more plurilingual background (students 1 and 3) learn to use codeswitching and translanguaging also strategically to express their multilingual personae and to regulate discourse. All students develop an increased awareness for the needs of the interlocutors in plurilingual settings and try to adapt their language production to them. 

New learning paths are discovered along this way. Compensatory strategies such as codeswitching are used in difficult linguistic situations with a scaffolding function, thus supporting the learners in their attempt to approach their own ZPD (zone of proximal development Vygozky). The contrastive use of languages, promoted by the plurilingual inputs, induces the learner to reflect on their own personal language production, and to critically monitor and correct it where necessary. This way, their proficiency in different languages is perceived more clearly, and their specific needs can be identified. Thanks to the self-regulated forms of learning, the necessary steps are taken by the individual learner to comply with the identified shortcomings. 

The activation of the multilingual repertoire through the simultaneous use of more languages accelerates with time in all students. This means that the languages are more easily retrievable, and that new ways of learning based on the interaction between languages can be found. Linguistic hierarchies are thereby laid bare and recognized as such, and the students develop a consciousness for their own emotional approach to the languages in question. They realize that each language is associated with particular emotions and that these emotions tend to depend, to a large extent, on each student’s own language background, influencing the way each language is used.


\section*{Abbreviations}
\label{sec:8:abbreviations}

\begin{tabular}{@{}ll}
CLIN & Crosslinguistic interaction\\
TBLL & Task-based language learning\\
FLT & Foreign-language teaching\\
TBLT & Task-based language teaching\\
SR & Simulated recalls\\
ZPD & Zone of proximal development Vygozky\\
\end{tabular}

{\sloppy\printbibliography[heading=subbibliography,notkeyword=this]}
\end{document}
